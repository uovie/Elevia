\documentclass[11pt,a4paper]{article}

\usepackage{geometry}
\usepackage{fontspec}
\usepackage{graphicx}
\usepackage{amsmath}
\usepackage{amssymb}
\usepackage{braket}
\usepackage{IEEEtrantools}
\usepackage[T1]{fontenc}
\usepackage[cmintegrals]{newtxmath}
\usepackage{bm}
\usepackage{tabularx}
\usepackage{enumitem}
\usepackage{listings}

\geometry{a4paper,left=2.5cm,right=2.5cm,top=2.5cm,bottom=2.5cm}

\setmainfont{Times New Roman}
\setsansfont{DejaVu Sans}
\setmonofont{Latin Modern Mono}

% define the title
\title{Elevia Manual}
\author{Haoyu Lin (1801110300)}
\date{\today}

% set code format
\lstset{
	basicstyle          =   \sffamily,          % 基本代码风格
	keywordstyle        =   \bfseries,          % 关键字风格
	commentstyle        =   \rmfamily \itshape, % 注释的风格,斜体
	stringstyle         =   \ttfamily,  % 字符串风格
	flexiblecolumns,                % 别问为什么,加上这个
	numbers             =   left,   % 行号的位置在左边
	showspaces          =   false,  % 是否显示空格,显示了有点乱,所以不现实了
	numberstyle         =   \ttfamily,    % 行号的样式,小五号,tt等宽字体
	showstringspaces    =   false,
	captionpos          =   t,      % 这段代码的名字所呈现的位置,t指的是top上面
	frame               =   lrtb,   % 显示边框
}

\lstdefinestyle{text}{
	basicstyle      =   \ttfamily,
	breaklines      =   true,   % 自动换行,建议不要写太长的行
	columns         =   fixed,  % 如果不加这一句,字间距就不固定,很丑,必须加
	basewidth       =   0.5em,
}

\bibliographystyle{unsrt}

\begin{document}
	
	% generates the title
	\maketitle
	
	\section{Input File Format}
	
	Elevia assumes a new input format:
	
	\begin{center}
	\noindent\begin{tabular}{|l|}
		\hline
		\text{[method] [basis] [the number of atoms] [charge] [spin multiplicity]}\\
		\\
		\text{[atom symbol]\ \ \ \ [  Rx  ]\ \ \ \ [  Ry  ]\ \ \ \ [  Rz  ]}\\
		\text{[atom symbol]\ \ \ \ [  Rx  ]\ \ \ \ [  Ry  ]\ \ \ \ [  Rz  ]}\\
		\text{[atom symbol]\ \ \ \ [  Rx  ]\ \ \ \ [  Ry  ]\ \ \ \ [  Rz  ]}\\
		... ...\\
		\hline
	\end{tabular}
	\end{center}
	
	\noindent For instance, an input file for test is
	
	%\lstinputlisting[
	%style   = text,
	%caption = {\bf H2O_STO3G.vie},
	%label   = {H2O_STO3G.vie}
	%]{../test/H2O_STO3G/H2O_STO3G.vie}
	
	\section{Method}
	Only support RHF.
	
	\section{Basis Set}
	Currently supported basis sets:
	\begin{itemize}
		\item STO-3G
		\item 3-21G
		\item 6-31G
		\item 6-31G(d)
		\item 6-31G(d,p)
		\item 6-311G(d,p)
	\end{itemize}
	Basis must strictly be typed in these displayed forms, only for that could Elevia recognize. The source code about reading basis data (dat/basis/*.g94) into the program is included in "intro\verb|_|basis.cpp".
	
	
	\section{SCF Procedure}
	The SCF procedure is carried out as follows:
	\begin{enumerate}
		\item Specify a system (a set of nuclear coordinates $\{\mathbf{R}_a\}$, atomic numbers $\{Z_a\}$, and the total number of electrons $N$) and a basis set $\{\chi_s\}$. 
		\item Calculate all required molecular integrals $S_{rs}$, $h_{rs}$, and $(rp|sq)$, $(rp|qs)$.
		\item Diagonalize the overlap matrix $\mathbf{S}$ and obtain a transformation matrix $\mathbf{X} = \mathbf{U} \mathbf{s}^{-1/2}$. 
		\item Assume an initial bond-order matrix $\mathbf{P}$ (often in the first iteration, we
		put $\mathbf{P} = \mathbf{0}$, as if there were no electron repulsion).
		\item \label{five} Find the $\boldsymbol{\mathcal{F}}$ matrix using matrix $\mathbf{P}$.
		\item Calculate the transformed Fock matrix $\boldsymbol{\mathcal{F}}^{\prime} = \mathbf{X}^{\dagger} \boldsymbol{\mathcal{F}} \mathbf{X}$.
		\item Diagonalize $\boldsymbol{\mathcal{F}}^{\prime}$ to obtain $\mathbf{C}^{\prime}$ and $\boldsymbol{\epsilon}$. 
		\item Calculate $\mathbf{C} = \mathbf{X} \mathbf{C}^{\prime}$.
		\item Form a new $\mathbf{P}$ from $\mathbf{C}$.
		\item Calculate the total energy $E$.
		\item Determine whether the procedure has converged, i.e., determine whether the difference between two successive total energy is less than a threshold ($10^{-8}$). If the procedure has not converged, return to step (\ref{five}) with the new bond-order matrix. If the procedure has converged, then calculate and output the quantities of interest.
	\end{enumerate}
	
	\cite{IQC} \cite{MQC} \cite{ERCMICG}
	
	\bibliography{Elevia_Manual.bib}

\end{document}