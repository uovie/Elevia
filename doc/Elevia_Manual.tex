\documentclass[11pt,a4paper]{article}

\usepackage{geometry}
\usepackage{fontspec}
\usepackage{graphicx}
\usepackage{amsmath}
\usepackage{amssymb}
\usepackage{braket}
\usepackage{IEEEtrantools}
\usepackage[T1]{fontenc}
\usepackage[cmintegrals]{newtxmath}
\usepackage{bm}
\usepackage{tabularx}
\usepackage{enumitem}

\geometry{a4paper,left=2.5cm,right=2.5cm,top=2.5cm,bottom=2.5cm}

\setmainfont{Times New Roman}
\setsansfont{DejaVu Sans}
\setmonofont{Latin Modern Mono}

% define the title
\title{Elevia Manual}
\author{Haoyu Lin (1801110300)}
\date{\today}

\bibliographystyle{unsrt}

\begin{document}
	
	% generates the title
	\maketitle
	
	\section{Basis Set}
	All the basis sets in Elevia are sets of orbitals comprised of Gaussian-type functions. Currently, the program only supports six types of built-in basis sets, namely STO-3G, 3-21G, 6-31G, 6-31G(d), 6-31G(d,p), 6-311G(d,p). And they are all in Gaussian94 format.
	
	
	\section{The Hartree-Fock-Roothaan Method (SCF LCAO MO)}
	
	\subsection{The Self-Consistent Field Method}
	The SCF procedure is carried out as follows:
	\begin{enumerate}
		\item Specify a system (a set of nuclear coordinates $\{\mathbf{R}_a\}$, atomic numbers $\{Z_a\}$, and the total number of electrons $N$) and a basis set $\{\chi_s\}$. 
		\item Calculate all required molecular integrals $S_{rs}$, $h_{rs}$, and $(rp|sq)$, $(rp|qs)$.
		\item Diagonalize the overlap matrix $\mathbf{S}$ and obtain a transformation matrix $\mathbf{X} = \mathbf{U} \mathbf{s}^{-1/2}$. 
		\item Assume an initial bond-order matrix $\mathbf{P}$ (often in the first iteration, we
		put $\mathbf{P} = \mathbf{0}$, as if there were no electron repulsion).
		\item \label{five} Find the $\boldsymbol{\mathcal{F}}$ matrix using matrix $\mathbf{P}$.
		\item Calculate the transformed Fock matrix $\boldsymbol{\mathcal{F}}^{\prime} = \mathbf{X}^{\dagger} \boldsymbol{\mathcal{F}} \mathbf{X}$.
		\item Diagonalize $\boldsymbol{\mathcal{F}}^{\prime}$ to obtain $\mathbf{C}^{\prime}$ and $\boldsymbol{\epsilon}$. 
		\item Calculate $\mathbf{C} = \mathbf{X} \mathbf{C}^{\prime}$.
		\item Form a new $\mathbf{P}$ from $\mathbf{C}$.
		\item Calculate the total energy $E$.
		\item Determine whether the procedure has converged, i.e., determine whether the difference between two successive total energy is less than a threshold ($10^{-8}$). If the procedure has not converged, return to step (\ref{five}) with the new bond-order matrix. If the procedure has converged, then calculate and output the quantities of interest.
	\end{enumerate}
	
	
	\bibliography{Elevia_Manual.bib}

\end{document}