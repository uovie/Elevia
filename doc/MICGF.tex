\documentclass[11pt,a4paper]{article}

\usepackage{geometry}
\usepackage{fontspec}
\usepackage{graphicx}
\usepackage{amsmath}
\usepackage{amssymb}
\usepackage{braket}
\usepackage{IEEEtrantools}
\usepackage[T1]{fontenc}
\usepackage[cmintegrals]{newtxmath}
\usepackage{bm}
\usepackage{tabularx}

\geometry{a4paper,left=2.5cm,right=2.5cm,top=2.5cm,bottom=2.5cm}

\setmainfont{Times New Roman}
\setsansfont{DejaVu Sans}
\setmonofont{Latin Modern Mono}

% define the title
\title{Molecular Integrals over Cartesian Gaussian Functions}
\author{Haoyu Lin (1801110300)}
\date{\today}

\bibliographystyle{unsrt}

\begin{document}
	
	% generates the title
	\maketitle
	
	\section{Cartesian Gaussian Functions}
	We write an unnormalized primitive Cartesian Gaussian function centered at $mathbf{R}$ as
	\begin{equation}
		g^{\prime}(\mathbf{r}; \zeta, \mathbf{n}, \mathbf{R}) = \left(x-R_{x}\right)^{n_{x}} \left(y-R_{y}\right)^{n_{y}} \left(z-R_{z}\right)^{n_{z}} \exp \left[ -\zeta (\mathbf{r} - \mathbf{R})^2 \right],
	\end{equation}
	where $\mathbf{r}$ is the coordinate vector of the electron, $\zeta$ is the orbital exponent, and n is a set of non-negative integers. The sum of $n_{x}$, $n_{y}$, and $n_{z}$ will be denoted $\lambda(\mathbf{n})$ and be referred to as the angular momentum or orbital quantum number of the Gaussian function. The functions with $\lambda(\mathbf{n})$ equal to $0,1,2,\dots$, are referred to as $s,p,d,\dots$, respectively. A set of $(\lambda + 1) (\lambda + 2)/2$ functions at $\mathbf{R}$ associated with the same angular momentum $\lambda$ and orbital exponent $\zeta$ constitute a shell, and the functions in the shell are components of the shell. Some examples are listed in the Table \ref{table: 1}.
	\begin{table}[htbp]
		\caption{Some shells and their components} 
		\label{table: 1}
		\begin{center}
		\begin{tabular}{|c|c|c|}
			\hline
			$\lambda(\mathbf{n})$ & Shell & Components ($i,j = x,y,z$) \\ \hline
			0 & s & $\mathbf{0} = (0,0,0)$ \\ \hline
			1 & p & $\mathbf{1}^i = (\delta_{ix}, \delta_{iy}, \delta_{iz})$ \\ \hline
			2 & d & $\mathbf{1}^i + \mathbf{1}^j$ \\
			\hline
		\end{tabular} 
		\end{center}
	\end{table}
	\\And the normalization coefficient $N(\zeta, \mathbf{n})$ of $g^{\prime}(\mathbf{r}; \zeta, \mathbf{n}, \mathbf{R})$ can be obtained through the equation
	\begin{equation}
		\int_{-\infty}^{\infty} x^{n} e^{-\zeta x^{2}} d x = \left\{
		\begin{array}{ll}
			2^{-(n-1)/2} \zeta^{-(n+1)/2}(n-1)!! & {\text{    for odd } n,} \\
			2^{-n/2} \pi^{1/2} \zeta^{-(n+1)/2}(n-1)!! & {\text{    for even } n,}
		\end{array}\right.
		\label{eq: one-dimensional Gaussian integral}
	\end{equation}
	by virtue of which, we have
	\begin{equation}
		N(\zeta, \mathbf{n}) = \left(\frac{2}{\pi}\right)^{3 / 4} \frac{2^{\lambda(\mathbf{n})} \zeta^{(2 \lambda(\mathbf{n})+3) / 4}}{\left[\left(2 n_{x}-1\right) ! !\left(2 n_{y}-1\right) ! !\left(2 n_{z}-1\right) ! !\right]^{1 / 2}}.
	\end{equation}
	
	A contracted Gaussian function is just a linear combination of primitive Gaussians (also termed primitives) centered at the same center $\mathbf{A}$ and with the same momentum indices $\mathbf{n}$ but with different exponents $\zeta_i$:
	\begin{equation}
		g(\mathbf{r}; \boldsymbol{\zeta}, \mathbf{n}, \mathbf{c}, \mathbf{R}) = \left(x-R_{x}\right)^{n_{x}} \left(y-R_{y}\right)^{n_{y}} \left(z-R_{z}\right)^{n_{z}} \sum_{i=1}^{M} C_i \exp \left[ -\zeta_i (\mathbf{r} - \mathbf{R})^2 \right],
	\end{equation}
	where $C_i = c_i N(\zeta_i, \mathbf{n})$ is the normalization-including contraction coefficient, and $c_i$ is the corresponding contraction coefficient.
	
	
	\section{Features of Gaussian Functions}
	
	\subsection{Product of GFs}
	The GTOs have an outstanding feature (along with the square dependence in the exponent),
	which decides about their importance in quantum chemistry. The product of two Gaussian-type 1s orbitals (even if they have different centers) is a single Gaussian-type 1s orbital.
	\begin{equation}
		\exp \left[ -\zeta_a (\mathbf{r} - \mathbf{R}_a)^{2} \right] \exp \left[ -\zeta_b (\mathbf{r} - \mathbf{R}_b)^{2} \right] = N_{ab} \exp \left[-\zeta_{ab} (\mathbf{r}-\mathbf{R}_{ab})^{2}\right],
	\end{equation}
	with parameters
	\begin{IEEEeqnarray}{rCl}
		\zeta_{ab} &=& \zeta_a + \zeta_b, \nonumber \\
		\mathbf{R}_{ab} &=& (\zeta_a \mathbf{R}_a + \zeta_b \mathbf{R}_b) / \zeta_{ab}, \nonumber \\
		N_{ab} &=& \exp \left[\zeta_{ab} \mathbf{R}_{ab}^2 - \left(\zeta_a \mathbf{R}_a^2 + \zeta_b \mathbf{R}_b^2 \right)\right].
	\end{IEEEeqnarray}
	And multiplying recursively, three and higher-fold products are derived:
	\begin{equation}
		\exp \left[ -\zeta_a (\mathbf{r} - \mathbf{R}_a)^{2} \right] \exp \left[ -\zeta_b (\mathbf{r} - \mathbf{R}_b)^{2} \right] \exp \left[ -\zeta_c (\mathbf{r} - \mathbf{R}_c)^{2} \right] = N_{abc} \exp \left[-\zeta_{abc} (\mathbf{r}-\mathbf{R}_{abc})^{2}\right],
	\end{equation}
	with parameters
	\begin{IEEEeqnarray}{rCl}
		\zeta_{abc} &=& \zeta_a + \zeta_b + \zeta_c, \nonumber \\
		\mathbf{R}_{abc} &=& (\zeta_a \mathbf{R}_a + \zeta_b \mathbf{R}_b + \zeta_c \mathbf{R}_c) / \zeta_{abc}, \nonumber \\
		N_{abc} &=& \exp \left[\zeta_{abc} \mathbf{R}_{abc}^2 - \left(\zeta_a \mathbf{R}_a^2 + \zeta_b \mathbf{R}_b^2 + \zeta_c \mathbf{R}_c^2\right)\right],
	\end{IEEEeqnarray}
	and so forth.
	
	\subsection{Differential Relation}
	The Cartesian Gaussian functions satisfy the differential relation
	\begin{equation}
		\frac{\partial}{\partial R_i} g^{\prime}(\mathbf{r}; \zeta, \mathbf{n}, \mathbf{R}) = 2 \zeta g^{\prime}(\mathbf{r}; \zeta, \mathbf{n}+\mathbf{1}^i, \mathbf{R}) - n_i g^{\prime}(\mathbf{r}; \zeta, \mathbf{n}-\mathbf{1}^i, \mathbf{R}) \qquad (i = x,y,z),
		\label{eq: differential relation 1}
	\end{equation}
	In the Cartesian Gaussian function the nuclear coordinate $R_i$ always appears in the form of $r_i - R_i$. Therefore, differentiation with respect to $R_i$ can be replaced by that with respect to $r_i$:
	\begin{equation}
		\frac{\partial}{\partial r_i} g^{\prime}(\mathbf{r}; \zeta, \mathbf{n}, \mathbf{R}) = n_i g^{\prime}(\mathbf{r}; \zeta, \mathbf{n}-\mathbf{1}^i, \mathbf{R}) - 2 \zeta g^{\prime}(\mathbf{r}; \zeta, \mathbf{n}+\mathbf{1}^i, \mathbf{R}) \qquad (i = x,y,z).
		\label{eq: differential relation 2}
	\end{equation}
	
	
	\section{Three-Center Overlap Integrals}
	Three-center overlap integrals over unnormalized Cartesian Gaussian functions are of the form:
	\begin{equation}
		(\mathbf{a}|\mathbf{c}|\mathbf{b}) = \int \mathrm{d} \mathbf{r}\, g^{\prime}(\mathbf{r}; \zeta_a, \mathbf{a}, \mathbf{R}_a) g^{\prime}(\mathbf{r}; \zeta_c, \mathbf{c}, \mathbf{R}_c) g^{\prime}(\mathbf{r}; \zeta_b, \mathbf{b}, \mathbf{R}_b).
	\end{equation}
	According to Eq.\eqref{eq: differential relation 1}, the integral $(\mathbf{a} + \mathbf{1}^i|\mathbf{c}|\mathbf{b})$ can be decomposed as
	\begin{equation}
		(\mathbf{a} + \mathbf{1}^i|\mathbf{c}|\mathbf{b}) = \frac{1}{2 \zeta_a} \frac{\partial}{\partial R_{a,i}} (\mathbf{a}|\mathbf{c}|\mathbf{b}) - \frac{1}{2 \zeta_a} a_i (\mathbf{a} - \mathbf{1}^i|\mathbf{c}|\mathbf{b}).
		\label{eq: iterative 1}
	\end{equation}
	Here the integral $(\mathbf{a}|\mathbf{c}|\mathbf{b})$ can be factored as
	\begin{equation}
		(\mathbf{a}|\mathbf{c}|\mathbf{b}) = N_{abc} I_x (a_x,b_x,c_x) I_y (a_y,b_y,c_y) I_z (a_z,b_z,c_z),
	\end{equation}
	where
	\begin{IEEEeqnarray}{rCl}
		I_i (a_i,b_i,c_i) &=& \left(\frac{\pi}{\zeta_{abc}}\right)^{1/2} \underbrace{\sum_{\alpha_i=0}^{a_i} \sum_{\beta_i=0}^{b_i} \sum_{\gamma_i=0}^{c_i}} _{\alpha_i + \beta_i + \gamma_i = \text{even}}
		\begin{pmatrix} a_i \\ \alpha_i \end{pmatrix}
		\begin{pmatrix} b_i \\ \beta_i \end{pmatrix}
		\begin{pmatrix} c_i \\ \gamma_i \end{pmatrix} \nonumber \\
		&& \times (R_{abc,i} - R_{a,i})^{a_i-\alpha_i} (R_{abc,i} - R_{b,i})^{b_i-\beta_i} (R_{abc,i} - R_{c,i})^{c_i-\gamma_i} \frac{(\alpha_i + \beta_i + \gamma_i - 1)!!}{(2\zeta_{abc})^{\alpha_i + \beta_i + \gamma_i}}.
	\end{IEEEeqnarray}
	Differentiating $N_{abc}$ and $I_i (a_i,b_i,c_i)$ with respect to $R_{a,i}$, we have
	\begin{equation}
		\frac{1}{2 \zeta_a} \frac{\partial}{\partial R_{a,i}} N_{abc} = (R_{abc,i} - R_{a,i}) N_{abc},
		\label{eq: differential 1}
	\end{equation}
	and
	\begin{IEEEeqnarray}{rCl}
		\frac{1}{2 \zeta_a} \frac{\partial}{\partial R_{a,i}} I_i (a_i,b_i,c_i) &=& a_i \left[ \frac{1}{2\zeta_{abc}} - \frac{1}{2\zeta_a}\right] I_i (a_i - 1,b_i,c_i) \nonumber \\
		&& \negmedspace {} + b_i \frac{1}{2\zeta_{abc}} I_i (a_i,b_i - 1,c_i) + c_i \frac{1}{2\zeta_{abc}} I_i (a_i,b_i,c_i - 1).
		\label{eq: differential 2}
	\end{IEEEeqnarray}
	Substitution of Eqs. \eqref{eq: differential 1} and \eqref{eq: differential 2} into Eq.\eqref{eq: iterative 1} gives finally
	\begin{IEEEeqnarray}{rCl}
		(\mathbf{a} + \mathbf{1}^i|\mathbf{c}|\mathbf{b}) &=& (R_{abc,i} - R_{a,i}) (\mathbf{a}|\mathbf{c}|\mathbf{b}) + \frac{1}{2\zeta_{abc}} \left[ a_i(\mathbf{a} - \mathbf{1}^i|\mathbf{c}|\mathbf{b}) + b_i(\mathbf{a}|\mathbf{c}|\mathbf{b} - \mathbf{1}^i) + c_i(\mathbf{a}|\mathbf{c} - \mathbf{1}^i|\mathbf{b})\right]. \IEEEeqnarraynumspace
	\end{IEEEeqnarray}
	The integral over s-functions is given by
	\begin{equation}
		(\mathbf{0}^{a}|\mathbf{0}^{c}|\mathbf{0}^{b}) = \left(\frac{\pi}{\zeta_{abc}}\right)^{3/2} N_{abc} = \left(\frac{\zeta_{ab}}{\zeta_{abc}}\right)^{3/2} (\mathbf{0}^{a}|\mathbf{0}^{b}) \exp \left[ -\frac{\zeta_{ab} \zeta_{c}}{\zeta_{abc}} (\mathbf{R}_{ab} - \mathbf{R}_c)^2 \right],
		\label{eq: Three-center s-type overlap integrals}
	\end{equation}
	where $(\mathbf{0}^{a}|\mathbf{0}^{b})$ is the overlap integral between two s-functions centered at $\mathbf{R}_a$ and $\mathbf{R}_b$:
	\begin{equation}
		(\mathbf{0}^{a}|\mathbf{0}^{b}) = (\pi/\zeta)^{3/2} \exp \left[ -\frac{\zeta_a \zeta_b}{\zeta_{ab}} (\mathbf{R}_a - \mathbf{R}_b)^2 \right].
	\end{equation}
	
	
	\section{Electron Repulsion Integrals}
	For the electron repulsion integrals (ERI's) over unnormalized Cartesian Gatissian functions
	\begin{equation}
		(\mathbf{a}\mathbf{b}|\mathbf{c}\mathbf{d}) = \int \mathrm{d} \mathbf{r}_1 \mathrm{d} \mathbf{r}_2\, g^{\prime}(\mathbf{r}_1; \zeta_a, \mathbf{a}, \mathbf{R}_a) g^{\prime}(\mathbf{r}_1; \zeta_b, \mathbf{b}, \mathbf{R}_b) \left| \mathbf{r}_1 - \mathbf{r}_2 \right|^{-1} g^{\prime}(\mathbf{r}_2; \zeta_c, \mathbf{c}, \mathbf{R}_c) g^{\prime}(\mathbf{r}_2; \zeta_d, \mathbf{d}, \mathbf{R}_d),
	\end{equation}
	we may substitute the identity
	\begin{equation}
		\left| \mathbf{r}_1 - \mathbf{r}_2 \right|^{-1} = \frac{2}{\pi^{1/2}} \int_{0}^{\infty} \mathrm{d}u\, \exp \left[ -(\mathbf{r}_1 - \mathbf{r}_2)^2 u^2 \right],
	\end{equation}
	to obtain
	\begin{equation}
		(\mathbf{a}\mathbf{b}|\mathbf{c}\mathbf{d}) = \frac{2}{\pi^{1/2}} \int_{0}^{\infty} \mathrm{d}u\, (\mathbf{a}\mathbf{b}|u|\mathbf{c}\mathbf{d}),
	\end{equation}
	where
	\begin{equation}
		(\mathbf{a}\mathbf{b}|u|\mathbf{c}\mathbf{d}) = \int \mathrm{d} \mathbf{r}_2 g^{\prime}(\mathbf{r}_2; \zeta_c, \mathbf{c}, \mathbf{R}_c) g^{\prime}(\mathbf{r}_2; \zeta_d, \mathbf{d}, \mathbf{R}_d) (\mathbf{a}|\mathbf{0}^{r_2}|\mathbf{b})
		\label{eq: (ab|u|cd)}
	\end{equation}
	and
	\begin{equation}
		(\mathbf{a}|\mathbf{0}^{r_2}|\mathbf{b}) = \int \mathrm{d} \mathbf{r}_1 g^{\prime}(\mathbf{r}_1; \zeta_a, \mathbf{a}, \mathbf{R}_a) g^{\prime}(\mathbf{r}_1; \zeta_b, \mathbf{b}, \mathbf{R}_b) \exp \left[ -u^2 (\mathbf{r}_1 - \mathbf{r}_2)^2 \right].
		\label{eq: integrate with an s-type PGF}
	\end{equation}
	Since the exponential function in Eq.\eqref{eq: integrate with an s-type PGF} corresponds to an s-type Cartesian Gaussian function with the orbital exponent $u^2$ centered at $\mathbf{r}_2$, $(\mathbf{a}|\mathbf{0}^{r_2}|\mathbf{b})$ is just a three-center overlap integral. Using Eq.\eqref{eq: Three-center s-type overlap integrals}, we obtain
	\begin{IEEEeqnarray}{rCl}
		(\mathbf{a} + \mathbf{1}^i|\mathbf{0}^{r_2}|\mathbf{b}) &=& (R_{ab,i} - R_{a,i}) (\mathbf{a}|\mathbf{0}^{r_2}|\mathbf{b}) + \frac{1}{2\zeta_{ab}} \left(1-\frac{\rho}{\zeta_{ab}} \frac{u^{2}}{\rho+u^{2}}\right) \left[ a_i(\mathbf{a} - \mathbf{1}^i|\mathbf{0}^{r_2}|\mathbf{b}) + b_i(\mathbf{a}|\mathbf{0}^{r_2}|\mathbf{b} - \mathbf{1}^i) \right] \nonumber \\
		&&+ \frac{1}{\zeta_{ab}+\zeta_{cd}} \frac{u^{2}}{\rho+u^{2}} \left[ \zeta_{cd} (r_{2i}-R_{ab,i}) (\mathbf{a}|\mathbf{0}^{r_2}|\mathbf{b}) - u^2 (\mathbf{a}|\mathbf{1}^{1,r_2}|\mathbf{b})\right],
		\label{eq: (a+1|0|b)}
	\end{IEEEeqnarray}
	where we have made use of the relations
	\begin{equation}
		(\mathbf{a}|\mathbf{1}^{i,r_2}|\mathbf{b}) = -\frac{\zeta_{ab}}{\zeta_{ab}+u^2} (r_{2i}-R_{ab,i}) (\mathbf{a}|\mathbf{0}^{r_2}|\mathbf{b}) + \frac{1}{2(\zeta_{ab}+u^2)} \left[ a_i(\mathbf{a} - \mathbf{1}^i|\mathbf{0}^{r_2}|\mathbf{b}) + b_i(\mathbf{a}|\mathbf{0}^{r_2}|\mathbf{b} - \mathbf{1}^i) \right],
	\end{equation}
	and
	\begin{equation}
		\frac{1}{\zeta_{ab}+u^{2}} = \frac{1}{\zeta_{ab}} \left(1-\frac{\zeta_{ab}\rho}{\zeta_a \zeta_b} \frac{u^2}{\rho+u^2}\right) - \frac{1}{\zeta_{ab}+\eta} \frac{u^{2}}{\zeta_{ab}+u^{2}} \frac{u^{2}}{\rho+u^{2}}
	\end{equation}
	with the parameters $\zeta_{cd}$ and $\rho$ defined by
	\begin{equation}
		\zeta_{cd} = \zeta_c+\zeta_d \qquad \text{and} \qquad \rho = \frac{\zeta_{ab}\zeta_{cd}}{\zeta_{ab}+\zeta_{cd}}.
	\end{equation}
	The integration over $\mathbf{r}_2$ of the last term of Eq.\eqref{eq: (a+1|0|b)} multiplied by $g^{\prime}(\mathbf{r}_2; \zeta_c, \mathbf{c}, \mathbf{R}_c)$ and $g^{\prime}(\mathbf{r}_2; \zeta_d, \mathbf{d}, \mathbf{R}_d)$ can be rewritten as
	\begin{IEEEeqnarray}{rCl}
		&&-\frac{1}{\zeta_{ab}+\zeta_{cd}} \frac{u^{2}}{\rho+u^{2}} \int \mathrm{d} \mathbf{r}_2\, g^{\prime}(\mathbf{r}_2; \zeta_c, \mathbf{c}, \mathbf{R}_c) g^{\prime}(\mathbf{r}_2; \zeta_d, \mathbf{d}, \mathbf{R}_d) u^2 (\mathbf{a}|\mathbf{1}^{1,r_2}|\mathbf{b}) \nonumber \\
		&=& \frac{1}{\zeta_{ab}+\zeta_{cd}} \frac{u^{2}}{\rho+u^{2}} \int \mathrm{d} \mathbf{r}_2\, g^{\prime}(\mathbf{r}_2; \zeta_a, \mathbf{a}, \mathbf{R}_a) g^{\prime}(\mathbf{r}_2; \zeta_b, \mathbf{b}, \mathbf{R}_b) u^2 (\mathbf{c}|\mathbf{1}^{1,r_1}|\mathbf{d}).
		\label{eq: rewritten last term}
	\end{IEEEeqnarray}
	Multiplying the recurrence formula for $(\mathbf{c}|\mathbf{i}^{1,r_1}|\mathbf{d})$ by $\zeta_{cd} + u^2$ we find
	\begin{IEEEeqnarray}{rCl}
		u^2 (\mathbf{c}|\mathbf{1}^{1,r_1}|\mathbf{d}) &=& -\zeta_{cd}(r_{1i}-\mathbf{R}_{cd}) (\mathbf{c}|\mathbf{0}^{r_1}|\mathbf{d}) \nonumber \\
		&&+ \frac{1}{2} \left[ (\mathbf{c}-\mathbf{1}^{i}|\mathbf{1}^{1,r_1}|\mathbf{d}) + (\mathbf{c}|\mathbf{1}^{1,r_1}|\mathbf{d}-\mathbf{1}^{i}) \right] - \zeta_{cd}(\mathbf{c}|\mathbf{1}^{1,r_1}|\mathbf{d}),
		\label{eq: Multiplying the recurrence formula}
	\end{IEEEeqnarray}
	where
	\begin{equation}
		\mathbf{R}_{cd} = (\zeta_c \mathbf{R}_c + \zeta_d \mathbf{R}_d) / \zeta_{cd}.
	\end{equation}
	Referring to Eq.\eqref{eq: (ab|u|cd)}, and using Eqs. \eqref{eq: (a+1|0|b)}, \eqref{eq: rewritten last term}, and \eqref{eq: Multiplying the recurrence formula}, we arrive at
	\begin{IEEEeqnarray}{rCl}
		((\mathbf{a}+\mathbf{1}^{i}) \mathbf{b}|u| \mathbf{c} \mathbf{d}) &=& (R_{ab,i}-R_{a,i})(\mathbf{a} \mathbf{b}|u| \mathbf{c} \mathbf{d}) + (R_{abcd,i}-R_{a,i}) \frac{u^{2}}{\rho+u^{2}} (\mathbf{a} \mathbf{b}|u| \mathbf{c} \mathbf{d}) \nonumber \\
		&& +\frac{1}{2 \zeta_{ab}} \left(1-\frac{\rho}{\zeta_{ab}} \frac{u^{2}}{\rho+u^{2}}\right) \left[ a_i((\mathbf{a}-\mathbf{1}^{i}) \mathbf{b}|u| \mathbf{c} \mathbf{d}) + b_i(\mathbf{a} (\mathbf{b}-\mathbf{1}^i)|u| \mathbf{c} \mathbf{d}) \right] \nonumber \\
		&& +\frac{1}{2(\zeta_{ab}+\zeta_{cd})} \frac{u^{2}}{\rho+u^{2}} \left[ c_i(\mathbf{a} \mathbf{b}|u|(\mathbf{c}-\mathbf{1}^{i}) \mathbf{d}) + d_i(\mathbf{a} \mathbf{b}|u| \mathbf{c}\left(\mathbf{d}-\mathbf{1}^{i}\right))\right],
	\end{IEEEeqnarray}
	where
	\begin{equation}
		\mathbf{R}_{abcd} = \frac{\zeta_{ab}\mathbf{R}_{ab} + \zeta_{cd}\mathbf{R}_{cd}}{\zeta_{ab} + \zeta_{cd}}.
	\end{equation}
	
	\bibliography{Basis_Set.bib}

\end{document}